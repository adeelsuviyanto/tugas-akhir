\documentclass[main]{subfiles}
\begin{document}
\pagestyle{myheadings} %use this untuk menghilangkan header chapter
\chapter{TINJAUAN PUSTAKA}

\section{Prinsip Kerja Sistem}
Berdasarkan rumusan masalah yang telah dipaparkan pada bab 1, berikut adalah konsep prinsip kerja sistem yang dicanangkan:
\begin{enumerate}
	\item Terdapat 2 drone dan 1 \textit{Base Station} (BS), masing-masing menggunakan mikrokontroler ESP32 sebagai \textit{Communications Module} (CM) yang berkomunikasi satu sama lain menggunakan \textit{User Datagram Protocol} (UDP) pada jaringan 802.11 Ad-Hoc.
	\item CM Drone \textit{Victim Finder} (VF) menerima sinyal dari darat melalui \textit{remote control} dari BS yang berisikan sinyal \textit{trigger} melalui UDP.
	\item CM Drone VF mengaktifkan modul GPS dan mendata koordinat lokasi, ketinggian terbang drone, dan status baterai drone.
	\item CM mengirimkan data melalui UDP \textit{Multi-cast} kepada drone kedua dan BS.
	\item \textit{*Hapus jika tidak diimplementasikan langsung*} Drone 2 menerima data lokasi dari drone 1, kemudian menggunakan CM untuk mengendalikan drone menggunakan Tello SDK.
\end{enumerate}

\end{document}

%pengujian di gd N, ujung ke ujung, dari sama tinggi ke beda tinggi
%bahas bab2 termasuk UAV, UDP, 802.11, OSI Layer, ESP32
